\newcommand{\metavars}
{
\begin{itemize}
 \item Expressions or mappings: \e, \dd, \rr
 \item Field names: \f, \g
 \item Variables: \x
 \item Method names: \m
 \item Values: \vv
 \item Method declarations: \M, \N
 \item Type names: \C, \D
 \item Types: \I, \J, \K, \T, \U, \V, \W
 \item Non-{\txt typeOf} Types: \OO, \PP, \Q, \R, \SSS
 \item Types that are either {\txt Object} or a type application: \A, \B
 \item Ground types (contain no type variables): \GG
 \item Type variables: \X, \Y, \Z
 \item Mappings (field name $\mapsto$ value): $\Phi$
\end{itemize}
}

\newcommand{\unproved}
{
\myProof
Unproved.
}

\newcommand{\preservation}{
\begin{theorem}[Subject Reduction]
\mbox{}\\
If \eeproves{\typ{\e}{\SSS}} and \trans{\e}{\ep}
then \eeproves{\typ{\ep}{\T}} where {\proves \emptyset {\subtype{\T}{\SSS}}}.
\end{theorem}
}
\newcommand{\progress}
{
\begin{theorem}[Progress]
\mbox{}\\
If \eeproves{\typ{\e}{\SSS}} then one of the following holds:
\begin{itemize}
\item $\e = \mapping{\of}{\ov \vv}{\GG}$
\item \trans \e \ep
\item $\e = \cast{\SSS}{\ep}$ and $\eeproves{\typ{\ep}{\T}}$ and $\eproves{\notsubtype \T \SSS}$.
\end{itemize}
\end{theorem}
}
\newcommand{\soundness}
{
\begin{theorem}[Soundness]
If \typ{\e}{\SSS} then either
\begin{itemize}
\item $\e \transarrow  \mapping{\of}{\ov \vv}{\GG}$
\item $\e \transarrow \ep$ where $\ep$ is an invalid cast
\item $\e$ reduces infinitely.
\end{itemize}
\end{theorem}
}

\newcommand{\preservationProof}
{
\myProof
We prove this by structural induction on the derivation of \trans{\e}{\ep}.

\begin{description}

\proofcase{\rulename {R-Object}}

Immediate.

\proofcase{\rulename {R-New}}

Immediate from the premises of \rulename{T-New} and \rulename{R-New}.

\proofcase{\rulename {R-Class}}

Immediate from the premises of \\\rulename{WF-ClassDef} and \rulename{R-Class}.

\proofcase{\rulename {R-Cast}}

By \rulename{T-Cast}, $\e = {\cast{\T}{\Phi_{\GG}}}$ must have type
$\T$.  By \rulename{T-Mapping}, $\ep = {\Phi_{\GG}}$ must have type
$\GG$.  By hypothesis of the reduction rule, $\eproves {\subtype \GG
\T}$.

\proofcase{\rulename{R-Field}}

We know that $\e = \invoke {\mapping \of \od \GG} \f_i$ and that $\ep
= \dd_i$.  Since $\e$ must have been typed by \rulename{T-Field}, we
know that ${\fields \GG} = \oT\ \of$ and $\eeproves {\typ \e \T_i}$.
Further, since ${\mapping \of \od \GG}$ must have been typed by
\rulename{T-Mapping}, we know that ${\eeproves {\typ {\dd_i} \SSS}}$
where $\eproves {\subtype \SSS \T}$.

\proofcase{\rulename {R-Invk}}

We know that $\e = \invoke {\Phi_{\GG}} {\mof \dd}$ and that \\$\ep =
{\substituteTwoArg \ox \od \this {\Phi_{\GG}}} \e_0$. Further, $\e$
was typed by \rulename{T-Invk} to have type $\U$ where $\mtype \m \GG
= \arrType \oT \U$ and $\eeproves {\typ \od \oTp}$ and $\eproves {\subtype \oTp
\oT}$.  As a premise of \rulename{R-Invk}, we know that $\mbody \m \GG
= (\ox, \e_0')$.

By the lemma $mtype$ and $mbody$ agree, $\emptyset;{\typ{\ox}{\oT}},$\\
$\proves {{\typ{\this}{\GG}}} {\typ{\e_0'}{\Up}}$ where $\eproves
{\subtype \Up \U}$.  Then by the lemma Substitution Preserves Typing,
$\eeproves {\typ \ep \Upp}$ where $\ep = {\substituteTwoArg \ox \od
    \this {\Phi_{\GG}}} \e_0$ and $\eproves {\subtype \Upp \Up}$.

\proofcase{\rulename {C-Cast}}

Trivial, since $\eproves{\typ{\cast{\T}{\e}}{\T}}$ for any $\e$.

\proofcase{\rulename {C-Map}}

Immediate from the induction hypothesis and the transitivity of subtyping.

\proofcase{\rulename {C-New}}

Immediate from the induction hypothesis and the transitivity of subtyping.

\proofcase{\rulename {C-Arg}}

Immediate from the induction hypothesis and the transitivity of subtyping.

\proofcase{\rulename {C-Rcvr}}

We know that ${\e = {\invoke {\e_0} {\mof \dd}}}$ and $\ep = \invoke {\e_0'}
{\mof \dd}$. Further, $\e$ must have been typed by \rulename{T-Invk},
which means that $\eept {\e_0} \W$ for some ground type $\W$, and that
$\eept \od \oV$ and $\esub \oV \oU$, and also $\mtype \m \W = \arrType
\oU \T$.  By the induction hypothesis, $\eept {\e_0'} \Wp$ where $\esub
\Wp \W$.  Therefore, by lemma Subtyping Preserves Method Typing,
$\mtype \m \Wp = \arrType \oU \T$ and thus $\eept {\invoke {\e_0'}
 {\mof \dd}} \T$ by \rulename{T-Invk}.

\proofcase{\rulename {C-Field}}

If $\trans {\invoke \e \f_i} {\invoke \ep \f_i}$, then $\trans \e
\ep$.  Further, ${\invoke \e \f_i}$ must have been typed by
\rulename{T-Field} to have type $\T_i$.  Therefore, by the induction
hypothesis, $\eept \e \SSS$ and $\eept \ep \Sp$ where $\esub \SSS
\Sp$.  Then, by the lemma Fields are Preserved by Subtypes, $\fields
\Sp = \fields {\SSS}\ @\ \oF$ for some $\oF$, and by
\rulename{T-Field}, $\eept {\invoke \ep \f_i} \T_i$.

\end{description}
}

\newcommand{\progressProof}
{
\myProof  By induction over the derivation of $\eept \e \SSS$.

\begin{description}

\proofcase{\rulename{T-Var}} This is a contradiction, since $\e$ is ground.

\proofcase{\rulename{T-Class}} In this case $\e = \CSp$ where $\CSp$
is ground.  Then by the lemma Agreement of {\it field-vals} and {\it
  fields}, $\fieldVals{\typeof \CSp} = \Tfe$ for some $\oT$, $\of$,
and $\oe$.  Further, $\typeparams{\CSp} = \oX \superSym \oI \kindSym
  \oJ$ for some $\oX, \oI, \oJ$. Therefore, \rulename{R-Class}
applies and \\$\trans \SSS {\mapping \of {\SpforX \CSp} {\typeof
    \CSp}}$.

\proofcase{\rulename{T-Mapping}} Either $\e$ is already a value, or
$\e = \mapping \of \oe \GG$ where not all of the $\oe$ are values.
Then by the induction hypothesis, there is some $i$ such that either
$\trans {\e_i} {\ep_i}$, in which case \rulename{C-Mapping} applies,
or $\e_i$ contains a bad cast, and the case is complete.

\proofcase{\rulename{T-Cast}}  Here there are three cases:
\begin{itemize}
\item $\e = \cast \SSS \ep$ where $\ep$ is not a mapping.  Then \rulename{C-Cast} applies.
\item $\e = \cast \SSS {\Phi_{\T}}$ where ${\esub \T \SSS}$.  Then \rulename{R-Cast} applies.
\item $\e = \cast \SSS {\Phi_{\T}}$  where ${\proves \empty {\notsubtype \T \SSS}}$.  Then $\e$ is a bad cast.
\end{itemize}

\proofcase{\rulename{T-New}} From the antecedent of \rulename{T-New} the premise of \rulename{R-New} applies.

\proofcase{\rulename{T-Invk}}  Here there are two cases.
\begin{itemize}
\item $\e = \invoke \rr {\mof \dd}$ where $\rr$ is not a mapping.  Then by the induction hypothesis, either $\rr$ contains a bad cast or $\trans \rr \rr'$ and \rulename{C-Rcvr} applies.
\item $\e = \invoke {\Phi_{\T}} {\mof \dd}$  We know from the antecedent that \\$\mtype \m {\bound \T} = \arrType \oU \V$ and therefore $\mtype \m \T = \arrType \oU \V$ since $\T$ is ground.  Therefore, since {\it mbody} is defined everywhere {\it mtype} is defined, $\mbody \m \T = (\ox, \e_0)$ for some $\ox$ and $\e_0$.  Thus \rulename{R-Invk} applies.
\end{itemize}

\proofcase{\rulename{T-Field}} Here there are two cases, either the receiver is a mapping or not.  In the first, by the antecedent of the typing rule, we can lookup the field successfully and apply \rulename{R-Field}.  Otherwise, we can apply \rulename{C-Field}.

\end{description}

}

\newcommand{\substitutionPreservesTyping}{
\begin{lemma}[Substitution Preserves Typing]
\mbox{}\\
If $\dgprove {\typ \e \T}$ and $\Gamma = \typ{\ox}{\oS}$ and
${\proves {\Delta;\emptyset} {\typ {\od} {\oU}}}$ and ${\dproves {\subtype \oU \oS}}$
then ${\proves {\Delta;\emptyset} {\typ {{\substitute \ox \od} \e} \Tp}}$ where ${\dproves {\subtype{\Tp}{\T}}}$.
\end{lemma}
}

\newcommand{\substitutionPreservesTypingProof}{\unproved}

\newcommand{\methodTypingLemma}
{
\begin{lemma}[Subtyping Preserves Method Typing]
\mbox{}\\
If $\mtype \m \U = \arrType \oT \SSS$ and $\proves \emptyset {\subtype \V \U}$ then
$\mtype \m \V = \arrType \oT \SSS$.
\end{lemma}
}

\newcommand{\methodTypingProof}
{
\myProof By induction over the derivation of $\esub \V \U$.

\begin{description}
\proofcase{\rulename{S-Reflex}}

Trivial.

\proofcase{\rulename{S-Bound}}

Not applicable.

\proofcase{\rulename{S-Super}}

In this case, $\V = \CR$, where \\$CT(\C) = {\typeFront {\typeArgsBounds \C \X \I \J} \B \A}
\braces{... \oM}$ \\and $\U = \RforX \A$.  We need to show that $\mtype
\m \CR = \arrType \oT \SSS$.   By lemma $methods$ is well-defined, it
suffices to show that $\simpleMethod \in \methods \CR$.

Since we know that $\simpleMethod \in$ \\$\methods \U$, we proceed by
induction on the derivation of $\methods \U$.

\subcase{$\U = \Object$}
Impossible, since $\Object$ has no methods.

\subcasenl{$\U = \typeArgs \D \W$}\\
Then, $\U = \RforX \A$.  By \rulename{MethodsClass}, $\methods \V =
\methods \CR = (\RforX \oM)\ \cup\ \methods {\RforX \A} = (\RforX
\oM)\ \cup\ \methods \U$.  Therefore, any method in  $\methods \U$
must be in $\methods \V$.

\proofcase{\rulename{S-Typeof}}

Analogous to \rulename{S-Super}.

\proofcase{\rulename{S-Trans}}  Let the intermediate type be
$\T$. Then $\esub \V \T$ and $\esub \T \U$.  Thus, by the induction
hypothesis, $\mtype \m \V = \mtype \m \T = \mtype \m \U$.

\end{description}
}

\newcommand{\fieldsPreservedLemma}
{
\begin{lemma}[Fields are Preserved by Subtypes]
\mbox{}\\
If $\fields \U = \oF$ and ${\proves {\emptyset} {\subtype \V \U}}$ then $\exists\ \oG$ where $\fields \V = \oF\ @\ \oG$.
\end{lemma}
}

\newcommand{\fieldsPreservedProof}
{
\myProof
We prove this by induction over the derivation for $\fields \V$.

\begin{description}
\proofcase{$\V = \Object$}

Then $\U = \Object$ and $\emptyset = \emptyset\ @\ \emptyset$.

\proofcasenl{$\V = \CR$}

Then $CT(\C) = {\typeFront {\typeArgsBounds \C \X \I \J} \B \A}
\braces{... {\oH} ...}$

Continue by induction on the derivation of $\esub \V \U$.  The only
interesting cases are \rulename{S-Super} and \rulename{S-Trans}.


\subcase{\rulename {S-Super}}

$\RforX \A = \U$. Thus $\oG = \oH$.

\subcase{\rulename {S-Trans}}

Let $\T$ be the intermediate type. Then \\$\fields \V = {\fields \T}\ @\
{\ov {\HH'}}$ and $\fields \V = {\fields \T}\ @\ {\ov {\HH''}}$ by the
induction hypothesis.  Thus $\oG = {\ov {\HH'}}\ @\ {\ov {\HH''}}$.

\proofcasenl{$\V = \typeof{\CS}$}

Then $CT(\C) = {\typeFront {\typeArgsBounds \C \X \I \J} \B \A}
\braces{{\oW\ \oh\ \oe} ...}$.  We continue by induction over the
derivation of $\esub \V \U$.  The only complex cases are
\rulename{S-Kind} and \rulename{S-Trans}.

\subcase{\rulename{S-Trans}}
As above.
\newpage
\subcasenl{\rulename {S-Kind}}

$\SforX \A = \U$ Then $\fields \V = \SforX \oW\ \oh$.  By
well-formedness of $\C$, we know that $\fields \B \subseteq \oW\ \oh$.
Then $\SforX \fields \B \subseteq \SforX \oW\ \oh$, and $\fields
{\SforX \B} \subseteq$\\$\SforX \oW\ \oh$ by lemma Substitution
Distributes over $fields$.
\end{description}

}

\newcommand{\soundnessProofShort}
{
\myProof

Immediate from Subject Reduction and Progress.
}

\newcommand{\fieldsAgreeLemma}{
\begin{lemma}[Agreement of $fields$ and {\it field-vals}]
\mbox{}\\
If $\fields \GG = \Tf$ and $\fieldVals \GG = \oTp\ \ofp\ \oe$ then
$\oT = \oTp$, $\of = \ofp$ and $\eept \oe \oS$ where $\esub \oS \oT$.
\end{lemma}
}
\newcommand{\fieldsAgreeProof}{
\myProof The proof is exactly analogous to the proof of lemma {\it mtype} and {\it mbody} Agree.
}

\newcommand{\noBadCastLemma}
{
\begin{lemma}[Reduction Preserves Safety]
If $\eept \e \T$ and $\trans \e \ep$ then $\e$ contains a bad cast iff $\ep$ does.
\end{lemma}
}

\newcommand{\typeSubstSubtypeLemma}
{
\begin{lemma}[Type Substitution Preserves Subtyping]
If $\Delta = \oX \superSym \oS$ and $\dproves {\subtype \U \T}$
and $\esub \oSp {\SpforX \oS}$
then ${\esub{\SpforX \U} {\SpforX \T}}$.
\end{lemma}
}

\newcommand{\typeSubstSubtypeProof}
{
\myProof By induction over the derivation of $\dproves {\subtype \U \T}$.
\begin{description}

\proofcase{\rulename{S-Reflex}} Immediate.

\proofcase{\rulename{S-Trans}} Let $\V$ be the intermediate type.  By
the induction hypothesis, ${\esub{\SpforX \U} {\SpforX \V}}$ and
${\esub{\SpforX \V} {\SpforX \T}}$ and application of
\rulename{S-Trans} completes the case

\newcommand{\RforY}{\substitute {\overline {\Y}} {\overline {\R}}}

\proofcase{\rulename{S-Super}} In this case $\U = \CR$ and $CT(\C) =
{\typeDefShort {\typeArgsBounds \C \Y \I \J} \B \A}$ and $\RforY \A =
\T$.  We must show that $\esub {\SpforX \CR} {\SpforX {\RforY \A}}$.
But note that:

$\SpforX \CR = \typeArg \C {\SpforX {\ov \R}} = \typeArg \C {\SpforX {\RforY {\ov \Y}}} =
\typeArg \C {{\substitute {\ov \Y} {\SpforX {\ov \R}}} {\ov \Y}}$.

Also, ${\SpforX {\RforY \A}} = {\substitute {\ov \Y} {\SpforX {\ov \R}}} \A$.  So,
$\esub {\typeArg \C {\SpforX {\ov \R}}} {\substitute {\ov \Y} {\SpforX {\ov \R}}} \A$
by \rulename{S-Super}, completing the case.

\proofcase{\rulename{S-Kind}} Analogous to \rulename{S-Super}.

\proofcase{\rulename{S-Bound}} In this case, $\U = \X_i$ and $\T =
\SSS_i$.  Then ${\SpforX \U} = \Sp_i$ and $\SpforX \T = \SpforX
\SSS_i$.  We are given $\esub {\Sp_i} {\SpforX \SSS_i}$, completing the case.

\end{description}
}

\newcommand{\typeSubstTypeLemma}
{
\begin{lemma}[Type Substitution Preserves Typing]
If $\Delta = \oX \superSym \oS$ and $\dgprove {\typ \e \T}$ and $\esub
\oSp {\SpforX \oS}$ then ${\eept {\SpforX \e} {\SpforX \T}}$.

\end{lemma}
}

\newcommand{\typeSubstTypeProof}
{
\unproved
}

\newcommand{\substmbodyLemma}
{
\begin{lemma}[Substitution Preserves $mbody$ Soundness]
If $~\simpleMethod \okin\ \GG$ and $\mbody{\m}{\GG} = (\ox,\e_0)$ then
${\proves {\emptyset;{\typ {\this} {\GG}},{\typ{\ox}{\oT}}} {\typ{\e_0}{\Up}}}$ where ${\proves {\emptyset} {\subtype \Up {\SforX \U}}}$.
\end{lemma}
}
\newcommand{\substmbodyProof}{\unproved}

\newcommand{\extensionMethodOKLemma}{
\begin{lemma}[Extension Preserves Method OKness]
If $\M \okin\ \GG$ and $\super \Gp = \GG$ then $\M \okin\ \Gp$.
\end{lemma}
}

\newcommand{\methodsDefinedLemma}{
\begin{lemma}[$methods$ is well-defined]
If ${\methodDef \U \m \T \x {\e}} \in \methods \GG$ and ${\methodDef \Up \m \Tp \x {\e}} \in \methods \GG$
then $\T = \Tp$ and $\U = \Up$.
\end{lemma}
}

\newcommand{\methodsDefinedProof}
{
\myProof Immediate from the side constraint on class tables that no two methods share the same name.
}

\newcommand{\extensionMethodOKProof}{\unproved}

\newcommand{\mtypembodyLemma}
{
\label{lem:agree}
\begin{lemma}[$mtype$ and $mbody$ agree]
If ${\mbody \m \GG} = (\ox, \e)$ and ${\mtype \m \GG} = \arrType \oT \U$ then
$\proves {\emptyset;{\typ \this \GG}, {\typ{\oX}{\oT}}} {\typ \e \Up}$ where $\proves \emptyset {\subtype \Up \U}$.
\end{lemma}
}

\newcommand{\substOKLemma}
{
\begin{lemma}[Type Substitution Preserves Type OKness]
If $\proves {\oX \superSym \oG} \T \ok$ and $\esub \oH \oG$ and $\proves \emptyset {\oH \ok}$
then $\proves \emptyset {{{\substitute \oX \oH} \T} \ok}$.
\end{lemma}
}

\newcommand{\substOKProof}
{
\myProof By induction over the derivation of $\proves {\oX \superSym \oG} \T \ok$.

\begin{description}

\proofcase{\rulename{WF-Object}} Immediate.

\proofcase{\rulename{WF-TypeOf}} Immediate from the induction
hypothesis.

\proofcase{\rulename{WF-Var}} In this case, $\T = \X_i$ and
${\substitute \oX \oH} \X_i = \HH_i$.  And we are given that $\proves
\emptyset {\oH \ok}$.

\proofcase{\rulename{WF-Class}} Immediate from lemma Type
Substitution Preserves Subtyping and the induction hypothesis.

\end{description}
}

\newcommand{\substDistFieldsLemma}
{
\begin{lemma}[Substitution Distributes over $fields$]
$\RforX {\fields \GG} = \fields {\RforX \GG}$
\end{lemma}
}

\newcommand{\substDistFieldsProof}
{
\myProof By induction over the derivation of $\fields \GG$.

\begin{description}
\proofcase{\rulename{F-Object}}

Trivial.

\proofcase{\rulename{F-Class}}

\newcommand{\SforY}{\substitute {\overline {\Y}} {\overline {\SSS}}}

Let $\GG = \CS$  Then $CT(\C) = {\typeFront {\typeArgsBounds \C \Y \I \J} \B \A}
\braces{... \oW\ \og}$, $\fields {\SforY A} = \oU\ \oh$ and $\fields \CS
= \oU\ \oh\ @\ {\SforY {\oW\ \og}}$.

Thus $\RforX \fields \CS = \RforX (\oU\ \oh\ @\ \SforY \oW\ \og) =
(\RforX \oU\ \oh)\ @\ (\RforX \SforY \oW\ \og) =
\fields {\RforX \CS}$.

\proofcase{\rulename{F-Typeof}}

In this case, $\GG = \typeof \CS$.  Then $CT(\C) = {\typeFront
{\typeArgsBounds \C \Y \I \J} \B \A} \braces{{\Tfe} ...}$ and $\fields
{\typeof \CS} = {\SforY}\Tf$. Therefore, $\RforX \fields {\typeof \CS}
= \RforX {\SforY}\Tf$.  But note that $\fields {\RforX {\typeof \CS}}
= \fields {\typeof {\typeArgs \C {\RforX \oS}}} = {\substitute \oY
{\RforX \oS}}\Tf = \RforX {\SforY}\Tf$.

\end{description}

}
