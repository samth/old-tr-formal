\documentclass{article}[12pt]
\usepackage{mmm}
\usepackage{sample}
\usepackage{amsmath}
\usepackage{amsthm}
\usepackage{fullpage}
\newtheorem{theorem}{Theorem}
\newtheorem{lemma}{Lemma}
\newtheorem{define}{Definition}

\newcommand\red{\rightarrow^*} 


\begin{document}

\title{Assignment 4}
\author{Sam Tobin-Hochstadt}

\newmeta\e{e}
\newmeta\vv{v}
\newmeta\kk{k}
%\newcommand\v{\vv{}}

\maketitle

\begin{theorem}
  If $\e{} \Downarrow v$ then $<e,E> \rightarrow^* <v,E>$.
\end{theorem}

\begin{proof}
  We proceed by induction the derivation of $\e{} \Downarrow v$ and by
  cases on the last rule of the derivation.  


\item[[$v \Downarrow v$]] Then $<e,E> \rightarrow^* <v,E>$ in zero steps.

\item[[$\beta$-reduction]] In this case $e = (\e1\ \e2)$.  Then by focus-down,
  $<e,E> \rightarrow^* <\e1,E[([]\ \e2)]>$.  Then by the induction
  hypothesis, this reduces to $<\lambda x.e',E[([]\ \e2)]>$.  Then we
  can apply focus-up to get $<(\lambda x.e'\ \e2),E>$, which reduces
  by the $\beta$-reduction rule to $<e'[\e2/x],E>$, which by the
  induction hypothesis reduces to $<v,E>$.  

\item[[if true]] In this case $e = (\e1\ \e2)$.  Then by focus-down,
  $<e,E> \rightarrow^* <\e1,E[([]\ \e2)]>$. By the induction
  hypothesis, this reduce to $<if true e',E[([]\ \e2)]>$.  By applying
  focus-up, we get $<if true e' \e2,E>$, which takes a step to
  $<e',E>$.  By the induction hypothesis, this reduces to $<v,E>$.  

\item[[if false]] This is exactly analagous to the above case.

\item[[zero?]] In this case, $e = (\e1\ \e2)$.  By focus-down, we have
  $<e,E> \rightarrow^* <\e1,E[([]\ \e2)]>$. By the induction
  hypothesis, we then have $<\e1,E[([]\ \e2)]> \red <zero?,E[([]\
  \e2)]>$.  Then by focus-up and focus-down, we have $<zero?,E[([]\
  \e2)]> \red <\e2,E[(zero?\ [])]>$.  By the induction hypothesis and
  focus-up, this reduces to $<(zero?\ \kk{m},E>$ which reduces to
  either true or false.  Since both true and false are values, we now
  have $<v,E>$ as desired.

\item[[+1, -1]] These cases are exactly analagous to zero?.

\item[[if]] In this case, $e = (\e1\ \e2)$.  By focus-down, we have
  $<e,E> \rightarrow^* <\e1,E[([]\ \e2)]>$. By the induction
  hypothesis, we then have $<\e1,E[([]\ \e2)]> \red <if,E[([]\
  \e2)]>$. Then by focus-up and focus-down, we have $<if,E[([]\
  \e2)]> \red <\e2,E[(if\ [])]>$.  By the induction hypothesis, this
  reduces to $<v',E[(if\ [])]>$.  By applying focus-up, we get $<if
  v,E>$, which is a value.

\item[[Other cases for if]] The other case for if proceeds as above.

\item[[Y]] In this case $e = (\e1\ \e2)$.  Then by focus-down,
  $<e,E> \rightarrow^* <\e1,E[([]\ \e2)]>$.  Then by the induction
  hypothesis, this reduces to $<Y,E[([]\ \e2)]>$.  By applying
  focus-up, we then get $<(Y\ \e2),E>$ which reduces by the reduction
  rule for $Y$ to $<(\e2\ (Y\ \e2)),E>$.  By the induction hypothesis,
  this reduces to $<v,E>$.  

\end{proof}

\end{document}