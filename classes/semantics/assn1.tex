\documentclass{article}


\usepackage{amssymb}
\usepackage{amsmath}
\usepackage{amsthm}
\usepackage{graphicx}

\newtheorem{theorem}{Theorem}
\newtheorem{lemma}{Lemma}
\newtheorem{define}{Definition}
\newcommand{\proofcase}[1]{\item[Case {\textsc {#1}}:]}


\begin{document}

\title{Assignment 1}
\author{Sam Tobin-Hochstadt}

\maketitle

\begin{lemma}[Substitution Preserves Typing]\mbox{}\\
If $e_1 : t'$ and $e_2 : t$ then $e_1[x^t/e_2]:t$.
\end{lemma}

\begin{proof}
  We proceed by structural induction and then by cases on the final
  rule in the derivation of $e_1 : t'$.

\begin{itemize}
  \proofcase{T-App} In this case, $e_1 = (e\ e')$.  Therefore
  $e_1[x^t/e_2] = (e\ e')[x^t/e_2] = (e[x^t/e_2]\ e'[x^t/e_2])$.  By the
  induction hypothesis, if $(e\ e') : t'$ then $(e[x^t/e_2]\
  e'[x^t/e_2]) : t'$.
  \proofcase{T-Abs} In this case, $e_1 = \lambda y.e$.  Then there are
  two cases.  If $y \neq x$, then substitution has no effect, and the
  statement is trivially true.  Otherwise, by the induction
  hypothesis, if $e : s$ then $e[e_2/x^t] : s$, and the premises of
  the rule are still satisfied.
  \proofcase{T-Const}  Here, substitution has no effect, and thus the
  lemma is trivially true.  
  \proofcase{T-Var}  Here, $e_1 = y$.  There are two cases.  If $y
  \neq x$, then substitution has no effect, and we are done.
  Otherwise, $y = x$, and $t = t'$, and therefore $e_1[x^t/e_2] = y$,
  and $y : t'$.  
\end{itemize}

\end{proof}

If $e_0 = ((\lambda x^t.e_1)\ e_2) : t'$, then the desired theorem is

\begin{theorem}[$\beta$-reduction Preserves Typing] \mbox{}\\
If $((\lambda x^t.e_1)\ e_2) : t'$ then $e_1[x^t/e_2]:t$.
\end{theorem}

To prove this, simply consider the typing rules for application and abstraction, which
shows that the abstraction must have type $t \rightarrow t'$, and therefore
$e_1 : t'$ and $e_2 : t$, from the typing of the application.  Thus we
can apply our lemma to obtain the desired conclusion.

\end{document}