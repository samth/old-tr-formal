\documentclass{amsart}


\usepackage{graphicx}

\newtheorem{theorem}{Theorem}
\newtheorem{lemma}{Lemma}
\newtheorem{define}{Definition}
\newcommand{\proofcase}[1]{\item[Case {\textsc {#1}}:]}


\begin{document}

\title{Assignment 2}
\author{Sam Tobin-Hochstadt}

\maketitle

\begin{define}
We define $E_1 \circ E_2$ inductively on the structure of $E_1$.
\end{define}

\begin{enumerate}
\proofcase{$E_1 = []$} Then $E_1 \circ E_2 = E_2$.  \vspace{2mm}

\proofcase{$E_1 = (E'_1\ e)$} Then $E_1 \circ E_2 = (E'_1 \circ E_2\
e)$. \vspace{2mm}

\proofcase{$E_1 = (op\ E'_1)$} Then $E_1 \circ E_2 = (op\ E'_1 \circ
E_2)$. \vspace{2mm}

\end{enumerate}

\begin{lemma}
$(E_1 \circ E_2) \circ E_3 = E_1 \circ (E_2 \circ E_3)$
\end{lemma}

\begin{proof}
We proceed by structural induction on $E_1$.

\proofcase{$E_1 = []$} \ \\
Then we have $([] \circ E_2) \circ E_3 = E_2 \circ E_3 = [] \circ (E_2
\circ E_3) = E_1 \circ (E_2 \circ E_3)$, since $[]$ is a left and
right unit for composition of contexts.
\proofcase{$E_1 = E'_1 \circ ([]\ e)$}\ \\
Then $ E_1 \circ (E_2 \circ E_3) = (E'_1 \circ ([]\ e)) \circ (E_2
\circ E_3)$. By applying the induction hypothesis at $E'_1$, this is
the same as $E'_1 \circ (([]\ e) \circ (E_2 \circ E_3)) = ((E'_1 \circ
([]\ e)) \circ E_2) \circ E_3$.  Then, by the defintion of $E_1$, this
is the same as $(E_1 \circ E_2) \circ E_3$.  


\proofcase{$E_1 = E'_1 \circ (op\ [])$}\ \\ Identical to the
above case, with $(op\ [])$ substitued for $([]\ e)$.
\end{proof}

\begin{theorem}
  $<E_1[q], E> \rightarrow^* <q,E \circ E_1>$
\end{theorem}

\begin{proof}
  We proceed by cases and structural induction on $E_1$.  If $E_1 =
  []$, then clearly both sides are equivalent to $<q,E>$, and the
  reduction is complete with 0 steps.  

  If $E_1 = (E'_1 e)$, then by the focus down rule, $<E_1[q], E>
  \rightarrow <E'_1[q], E \circ ([]\ e)>$. By induction, this reduces
  to $<q,(E \circ ([]\ e)) \circ E'_1>$ in some number of steps.  By
  Lemma 1, this is equivalent to $<q,E \circ (([]\ e) \circ E'_1)>$,
  which is the same as $<q,E \circ E_1>$.

  An identical argument applies if $E_1 = (op E'_1)$.
\end{proof}

\end{document}