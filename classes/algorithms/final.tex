\documentclass[twoside]{amsart}
\usepackage{geometry}
\usepackage{stmaryrd}
\usepackage{cancel}
\usepackage{clrscode}
\usepackage{textcomp}
\geometry{tmargin=1in,bmargin=1in,lmargin=1in,rmargin=1in}
\usepackage[curve]{xypic}

\newsavebox{\fmbox}
\newenvironment{fmpage}[1]
{\begin{lrbox}{\fmbox}
   \begin{minipage}{#1}}
{\end{minipage}
\end{lrbox}\fbox{\usebox{\fmbox}}}

\newenvironment{directive}{\begin{fmpage}{5.5in}}{\end{fmpage}}
\newenvironment{problem}[1]{\newpage\section{#1}\begin{fmpage}{5.5in}}{\end{fmpage}}

\newtheorem{prop}{Proposition}
\newtheorem{claim}[prop]{Claim}
\newtheorem{lemma}[prop]{Lemma}

\begin{document}

{\raggedleft
Sam Tobin-Hochstadt \\
CSG 713 Final Exam \\
\today \\
{\tt samth@ccs.neu.edu} \\
}

%\newcommand{\min}[#1]{\textrm{min}(

\section*{Problem 1}

First, initialize min($v_1$) to $v_1$.  Then, for each successive
vertex $v_k$, perform BFS from that vertex.  If a previously finished
vertex $v_j$ is reached, then terminate the BFS and min($v_k$) =
min($v_j$). This algorithm visits every vertex and every edge at most
once, and thus takes $O(V + E)$ time.

\section*{Problem 2}

\subsection*{Part i}
This conjecture is false.  No execution of Breadth-First search can
give the minimal spanning tree of this graph.

\xymatrix{
  v_1 \ar@{-}[d]_{100} \ar@{-}[r]^1 \ar@{-}[dr]^1 & v_2 \ar@{-}[d]^{100} \\
  v_3 \ar@{-}[r]^1 & v_4
}

Any execution of BFS must begin at one of the four vertices.  It must
then visit all of the edges incident on that vertex.  For any vertex,
that includes one of the edges weighted 100.  Thus, that execution
cannot produce the minimal spanning tree. \qed

\subsection*{Part ii}
This conjecture is false. No execution of Depth-First search can
produce the minimal spanning tree of this graph.

\xymatrix{
  v_1 \ar@{-}[rr]^{100} \ar@{-}[ddr]_{100} \ar@{-}[dr]^1  & \txt{} & v_2 \ar@{-}[ddl]^{100}
  \ar@{-}[dl]_1    \\
  \txt{} & v_3 \ar@{-}[d]^1   & \txt{} \\
  \txt{} & v_4    & \txt{} 
}

There are two cases.  One, the DFS starts at the center ($v_3$).  Then it
moves to one of the other vertices (WLOG $v_1$), and it encounters an
edge weighted 100 to $v_2$, which has not yet been visited.
Therefore, that edge will be included in the final spanning tree, and
the tree will not be minimal.

In the other case, the DFS starts on one of the edge vertices (WLOG
$v_1$).  Either it begins with one of the outer edges, in which case
the tree cannot be minimal.  Otherwise, it moves to the middle, and
then to one of the other outer vertices (WLOG $v_2$).  Then there is
an edge to $v_4$, which is not yet explored, so that edge is taken.
Therefore, the tree will not be minimal. \qed

\section*{Problem 3}

\subsection*{Part i}

This conjecture is true.

\subsection*{Part ii}

This conjecture is false. This graph

\xymatrix{
  v1 \ar@{-}[r]^2 \ar@{-}[d]_2 & v2 \\
  v3 \ar@{-}[r]_1 & v4 \ar@{-}[u]_1
}

has the following MMST:

\xymatrix{
  v1 \ar@{-}[r]^2 \ar@{-}[d]_2 & v2 \\
  v3  & v4 \ar@{-}[u]_1
}

but this is an MST with less total weight.

\xymatrix{
  v1 \ar@{-}[r]^2  & v2 \\
  v3 \ar@{-}[r]_1 & v4 \ar@{-}[u]_1
}

\qed

\section*{Problem 4}

The Laksurk method successfully finds a minimum spanning tree for any
graph.  


\section*{Problem 5}


\section*{Problem 6}

\subsection*{Part i}

\subsection*{Part ii}

Create a vertex for every number in the sequence.  Then create an edge
from every vertex to 

\section*{Problem 7}


\section*{Problem 8}

To see that an approximation ratio of 2 is tight, consider the
following graph:

\xymatrix{
  v_1 \ar@{-}[r] & v_3 \ar@{-}[r] & v_2
}

Then $A$ and $B$ will be initialized, and there will only be one
remaining vertex.  Since $d(v_3,\{v_1\}) = d(v_3,\{v_2\}) = 1$, $v_3$
will be allocated to $B$.  This produces a cut of size 1.  However,
the maximal cut is $(\{v_1,v_2\},\{v_3\})$, which has size 2.

\end{document}
