\documentclass{article}[12pt]

\usepackage{ifthen}
\usepackage{mmm}
\usepackage{ts}
\usepackage{amsmath}
\usepackage{amsthm}
\usepackage{fullpage}
\newtheorem{theorem}{Theorem}
\newtheorem{lemma}{Lemma}
\newtheorem{define}{Definition}
\usepackage{mathpartir}
\usepackage{rotating}

\newcommand\red{\to^*} 
\newcommand{\sembrack}[1]{[\![#1]\!]}


\ifthenelse{\equal{\multiargs}{true}}{
\margtrue
}
{
\margfalse
}


\ifmarg
\title{Typed Scheme with Multiple Arguments}
\else
\title{Typed Scheme}
\fi
\date{\today}

\begin{document}


\maketitle

\section{Syntax}


\[
  \begin{altgrammar}
    \d{}, \e{}, \dots &::=& \x{}  \alt \comb{\e{op}}{\movervec{\e{a}}} 
    \alt \cond{\e1}{\e2}{\e3} \alt \cons{\e{}}{\e{}} \alt \vv{}  &\mbox{Expressions} \\
    \vv{} &::=& \c{}  \alt \tt \alt \ff \alt \n{} \alt \abs{\x{}}{\t{}}{\e{}} \alt \cons{\vv{}}{\vv{}}  & \mbox{Values} \\
    \c{} &::=& \addone \alt \zerop \alt \numberp \alt \boolp \alt \procp \alt \notsym \alt \\
    &   & \consp \alt \car \alt \cdr & \mbox{Primitive Operations} 
    \vspace{1mm}
    \\
    \E{} &::=& [] \alt 
    \marg
        {\comb{\overvec{\vv{}}}{\E{}\ {\overvec{\e{}}}}} % general multi-arg contexts
        {\comb{\E{}}{\e{}} \alt \comb{\vv{}}{\E{}}} % one-arg contexts
        \alt \cons{\E{}}{\e{}} \alt \cons{\vv{}}{\E{}} \alt \cond{\E{}}{\e{}}{\e{}} 
        & \mbox{Evaluation Contexts} \vspace{1mm} \\
  \end{altgrammar}
  \]
\hrule
\[
  \begin{altgrammar}
    \sig{},\t{} &::= & \top \alt \marg{\proctop \alt}{} \num \alt \ttt \alt \fft 
    \alt \consty{\t{}}{\t{}} \alt
         {\proctype {\movervec[i]{\t{}}} {\t{}} {\movervec[i]{\phih{}}} {\sh{}}} 
         \alt (\usym\ \overvec{\t{}})     &\mbox{Types} \\



         \p{} &::=&  \t{\pi(\x{})} \alt \compt{\pi(\x{})} \alt \bot & \mbox{Filters} \\

         \ph{} & ::=& \t{\pi} \alt \compt{\pi}  \alt \both & \mbox{Latent Filters} \\

         \phii{} & ::= & \overvec{\p{}}|\overvec{\p{}} & \mbox{Filter Sets} \\

         \phih{} & ::= & \overvec{\ph{}}|\overvec{\ph{}} & \mbox{Latent Filter Sets} \\

         \s{} & ::= &   \pi(\x{}) \alt \noeffect &\mbox{Objects} \\
         
         \sh{} & ::= & \pii \alt \noeffect  & \mbox{Projections (Latent Objects)} \\

         \pi & ::= & \overvec{\pe{}} & \mbox{Paths} \\

         \pe{} & :: = & \pecar \alt \pecdr & \mbox{Path Elements} \\
         
         \Gamma & ::= & \overvec{\x{}:\t{}} & \mbox{Type Environments} \\
  \end{altgrammar}
  \]

\paragraph{Notation}
We write \bool for $(\usym \ttt\ \fft)$ and $\bot$ for $(\usym)$ in
type contexts and $\bullet$ for the empty sequence.  We also
abbreviate $\bullet(\x{})$ as $\x{}$.  Sequences are assumed to be
automatically flattened, so $\overvec{\t1},\overvec{\t2}$ is a sequence 
of types with $|\overvec{\t1}| + |\overvec{\t2}|$ elements, not a two
element sequence of sequences. 
When $\bot$ is an element of a sequence, we
will treat it as absorbing all the other elements, so 
$\op1,\bot,\op2 = \bot$.
 Finally, note that if we have $\op{}|\bot$, $\op{}$
cannot contain any information we don't already know.  So we will
assume that $\op{} = \bullet$ (and similarly for $\bot|\op{}$).

\ifmarg
The type $\proctop$ represents the supertype of all procedures, so
$\subtype{\proctype {\movervec[i]{\t{}}}{\sig}{\movervec[i]{\phih{}}}
  {\sh{}}}{\proctop}$ for any $\movervec[i]{\t{}}$, \sig{},
${\movervec[i]{\phih{}}}$  and $\sh{}$.  Note that this type is not
otherwise representable.
\else
\fi

\section{Type Judgement}

\huge
\begin{displaymath}
  \hastyeffphi{\e{}}{\t{}}{\phii{}}{\s{}}
\end{displaymath}
\normalsize

This means that in environment $\Gamma$, \e{} has the type \t{}, that
evaluating \e{} and gives us information \phii{} based on whether \e{}
evaluates to \ff or not, and that testing \e{} is the same as testing
the object \s{}.

\newpage

\section{Type Rules}

\paragraph{Core Type Rules}

\[
\inferrule[T-Var]{}{\hastyvar {\x{}} {\G{}(\x{})} {\x{}}}
\qquad\qquad
\inferrule[T-Num]{}{\hastytrue{\n{}}{\num}} 
\qquad\qquad
\inferrule[T-Const]{}{\hastytrue{\c{}}{\dt{\c{}}}}
\]



\[
\inferrule[T-True]{}{\hastytrue{\tt}{\ttt}}
\qquad\qquad
\inferrule[T-False]{}{\hastyfalse{\ff}{\fft}}
\]


\newcommand{\msubi}[1]{\marg{{#1}_i}{#1}}

\renewcommand{\xi}{\msubi{\x{}}}
\newcommand{\sai}{\msubi{\s{a}}}


\[
\inferrule[T-Abs]
{\hastyeffphi [\G{},{\movervec[i]{\hastype{\x{}}{\sig{}}}}] {\e{}} {\t{}} {\phii{}} {\s{}} \\\\
\movervec[i]{\phih{} = \abstractfilter {\xi} {\phii{}}} \\
\sh{} = \left\{
 \protect{\begin{array}{ll}
  \pii     & \text{if } \s{} = \pi(\xi) \\
  \noeffect  & \text{otherwise}
 \end{array}} \right.
}
{\hastytrue
  {\abs[i]{\x{}}{\sig{}}{\e{}}} 
  {\proctype {\movervec[i]{\sig{}}} {\t{}} {\movervec[i]{\phih{}}} {\sh{}}}
}
\qquad\qquad
\inferrule[T-App]
{ \hastyeffphi {\e{op}} {\t{op}}   {\phii{op}} {\s{op}}\\  
  \movervec[i]{\hastyeffphi {\e{a}} {\t{a}}  {\phii{a}} {\s{a}}}\\\\ 
  \movervec[i]{\subtype {\t{a}}{\t{f}}} \\
  {\subtype {\t{op}} {\proctype{\movervec[i]{\t{f}}}{\t{r}}{\movervec[i]{\phih{f}}}{\sh{}}}} \\\\
\phii{r} = \movervec[i]{\applyfilter{\phih{f}}{\t{a}}{\s{a}}} \\\\
\s{r} = \left\{
  \protect{\begin{array}{ll}
      \pi'(\pi(x))     & \text{if } \sh{} = \pipi \text{ and } {\sai} = \pi(\x{}) \\
      \noeffect  & \text{otherwise}
    \end{array}} \right.
}
{\hastyeffphi {\comb{\e{op}}{\movervec[i]{\e{a}}}} {\t{r}} {\phii{r}} {\s{r}}}
\]

\[
\inferrule[T-If]
{
  \hastyeff{\e1}{\t1}{\op1}{\op2}{\s1}
  \\\\
  \hastyeffphi[\G{} + \op1]{\e2}{\t2}{\phii2} {\s2}
  \\\\
  \hastyeffphi[\G{} + \op2]{\e3}{\t3}{\phii3} {\s3}
  \\\\
  \subtype{\t2}{\t{}} \\
  \subtype{\t3}{\t{}} \\\\
  \phii{} = \combfilter{\op1|\op2}{\phii2}{\phii3}
}
{\hastyeffphi {\cond{\e1}{\e2}{\e3}} {\t{}}  {\phii{}} {\noeffect}}
\]

\paragraph{Rules for Pairs}

\begin{displaymath}
  \inferrule[T-Cons]
  {
    \hastyeffphi{\e1}{\t1}{\phii1}{\s1} \\\\ \hastyeffphi{\e2}{\t2}{\phii2}{\s2} 
  }
  {
    \hastytrue{\cons{\e1}{\e2}} {\consty{\t1}{\t2}}
  }
\end{displaymath}

\begin{displaymath}
\inferrule[T-Car]
  {
    \hastyeffphi{\e{}} {\consty{\t1}{\t2}} {\phii{}} {\s{}} \\\\
    \phii{r} = \applyfilter{\comp{\fft}_{\pecar}|{\fft}_{\pecar}}{\consty{\t1}{\t2}} {\s{}} \\\\
    \s{r} = \left\{
      \protect{\begin{array}{ll}
          \pecar(\pi(x))     & \text{if } {\s{}} = \pi(\x{}) \\
          \noeffect  & \text{otherwise}
        \end{array}} \right.
  }
  {
    \hastyeffphi{\comb{\car}{\e{}}}{\t1}{\phii{r}}{\s{r}}
  }
  \qquad\qquad
  \inferrule[T-Cdr]
  {
    \hastyeffphi{\e{}} {\consty{\t1}{\t2}} {\phii{}} {\s{}} \\\\
    \phii{r} = \applyfilter{\comp{\fft}_{\pecdr}|{\fft}_{\pecdr}}{\consty{\t1}{\t2}} {\s{}} \\\\
    \s{r} = \left\{
      \protect{\begin{array}{ll}
          \pecdr(\pi(x))     & \text{if } {\s{}} = \pi(\x{}) \\
          \noeffect  & \text{otherwise}
        \end{array}} \right.
  }
  {
    \hastyeffphi{\comb{\cdr}{\e{}}}{\t2}{\phii{r}}{\s{r}} 
  }
\end{displaymath}

\paragraph{Auxilliary Type Rules}

\begin{displaymath}
  \inferrule[T-Not]
  {
    \hastyeff{\e{}}{\t{}}{\op1}{\op2}{\s{}}      
  }
  {
    \hastyeff{\nott{\e{}}}{\bool}{\op2}{\op1}{\noeffect}      
  }
  \qquad\qquad
  \inferrule[T-Bot]
  {
    \Gamma(\x{}) = \bot \\ \x{} \in \mbox{fv}(\e{}) 
  }
  {
    \hastyeffphi{\e{}}{\bot}{\bullet|\bullet}{\noeffect}
  }
\end{displaymath}

{\sc T-Bot} is needed to avoid complaints about spurious unreachable code.

\paragraph{Proof-theoretic Type Rules}

\[
\inferrule[T-IfTrue]
{
  \hastyeff{\e1}{\t1}{\bullet}{\bot}{\s1}
  \\\\
  \hastyeffphi[\G{} + \op1]{\e2}{\t2}{\phii2} {\s2}
  \\\\
  \subtype{\t2}{\t{}} \\\\
  \phii{} = \combfilter{\bullet|\bot}{\phii2}{\bullet|\bullet}
}
{\hastyeffphi {\cond{\e1}{\e2}{\e3}} {\t{}}  {\phii{}} {\noeffect}}
\qquad\qquad
\inferrule[T-IfFalse]
{
  \hastyeff{\e1}{\t1}{\bot}{\bullet}{\s1}
  \\\\
  \hastyeffphi[\G{} + \op2]{\e3}{\t3}{\phii3} {\s3}
  \\\\
  \subtype{\t3}{\t{}} \\\\
  \phii{} = \combfilter{\bot|\bullet}{\bullet|\bullet}{\phii3}
}
{\hastyeffphi {\cond{\e1}{\e2}{\e3}} {\t{}}  {\phii{}} {\noeffect}}
\]

\newpage

\section{Subtyping}


\[
\inferrule[S-Refl]{ }{\subtype{\t{}}{\t{}}}
\ifmarg
\qquad\qquad
\inferrule[S-ProcTop]
{ }{\subtype {\proctype {\movervec[i]{\sig{}}} {\t{}} {\movervec[i]{\phih{}}} {\sh{}}} {\proctop}}
\else
\fi
\qquad\qquad
\inferrule[S-Top]{ }{\subtype{\t{}}{\top}}
\]

\[
\inferrule[S-Pair]
{
  \subtype{\t1}{\t2} \\
  \subtype{\sig1}{\sig2}
}
{
  \subtype{\consty{\t1}{\sig1}}{\consty{\t2}{\sig2}}
}
\qquad\qquad
\inferrule[S-Fun]
{
  \movervec[i]{\subtype{\sig{a}}{\t{a}}} \\ 
  \subtype{\t{r}}{\sig{r}} \\\\ 
  \movervec[i]{\phih{}' \subseteq \phih{}} \\ \sh{} = \sh{}' \text{ or } \sh{}' = \noeffect
}
{
  \subtype
  {\proctype{\movervec[i]{\t{a}}}{\t{r}}{\movervec[i]{\phih{}}}{\sh{}}}
  {\proctype{\movervec[i]{\sig{a}}}{\sig{r}}{\movervec[i]{\phih{}'}}{\sh{}'}}
}
\]

\[
\inferrule[S-UnionSuper]
{\exists i.\ {\subtype{\t{}}{\sig{i}}}}
{\subtype{\t{}}{(\usym\ \overvec[i]{\sig{}})}}
\qquad\qquad
\inferrule[S-UnionSub]
{\overvec[i]{\subtype{\t{i}}{\sig{}}}}
{\subtype{(\usym\ \overvec[i]{\t{}})}{\sig{}}}
\]

\vspace{5mm}

There's still a problem here: 
\notsubtype 
{\consty{\uniontwo{\t{}}{\sig{}}}{\ups{}}}
{\uniontwo {\consty{\t{}} {\ups{}}} {\consty{\sig{}} {\ups{}}}  } 

I don't know how to fix this at the moment, short of a \usym-normalization step.

%% Unclear if we need a subeffecting relation - it was only used in the proof

% \[
% \inferrule[SE-Refl]{}{\subeff{\p{}}{\p{}}}
% \qquad\qquad
% \inferrule[SE-None]{}{\subeff{\p{}}{\noeffect}}
% \]

% \[
% \inferrule[SE-True]{\p{} \neq \ff}{\subeff{\tt}{\p{}}}
% \qquad\qquad
% \inferrule[SE-False]{\p{} \neq \tt}{\subeff{\ff}{\p{}}}
% \]


\newpage

\section{Metafunctions}

\newcommand{\af}[3]{\abstractfilter{#1}{#2} & {#3}}
\newcommand{\apf}[4]{\applyfilter{#1}{#2}{#3} & {#4}}

\subsection{Abstract}

$$
  \begin{array}{l@{\ =\ }l@{\qquad}l}
    \af{\x{}}{\op1|\op2}{\overvec{\abstractone{\x{}}{\p1}}|\overvec{\abstractone{\x{}}{\p2}}} &  \vspace{2mm}\\

    \abstractone{\x{}}{\bot}& {\both} & \\
    \abstractone{\x{}}{{\t{}}_{\pi(\x{})}}& {{\t{}}_{\pi}} & \\
    \abstractone{\x{}}{\comp{\t{}}_{\pi(\x{})}}& {\comp{\t{}}_{\pi}} & \\
    \abstractone{\x{}}{{\t{}}_{\pi(\y{})}}& {\bullet} &  \text{ where } \x{} \neq \y{} \\
    \abstractone{\x{}}{\comp{\t{}}_{\pi(\y{})}}& {\bullet} & \text{ where } \x{} \neq \y{} \\
%    \text{ where }
%    \ph{} = \left\{
%      \protect{\begin{array}{ll}
%          \both             & \text{if } \bot \in \op{i} \\
%          \bullet        & \text{otherwise}
%        \end{array}} \right.
   \end{array}
$$
%
\subsection{Apply}


$$
  \begin{array}{l@{\ =\ }l@{\qquad}l}
    \apf{\oph{+}|\oph{-}}{\t{}}{\s{}}{\overvec{\applyone{\ph+}{\t{}}{\s{}}}|\overvec{\applyone{\ph-}{\t{}}{\s{}}}} &
%    {\overvec[i]{\p{+_+}},\overvec[j]{\p{-_-}}|
%     \overvec[i]{\p{-_+}},\overvec[j]{\p{+_-}}} & 
%   \text{ where } \overvec[i]{\p{+_+},{\p{+_-}}}  = \overvec[i]{\applyone{\ph+}{\t{}}{\s{}}} \\
%   & & & \text{ and } \overvec[j]{\p{-_+},{\p{-_-}}}  = \overvec[j]{\applyone{\ph-}{\t{}}{\s{}}} \\
\vspace{2mm}\\



    \applyone{\both}{\sig{}}{\s{}} & \bot & \\
    \applyone{\comp{\t{}}_{\bullet}} {\sig{}} {\s{}} & \bot & \text{ where } \subtype{\sig{}}{\t{}}\\
    \applyone{{\t{}}_{\bullet}} {\sig{}} {\s{}} & \bot & \text{ where } \nooverlap{\sig{}}{\t{}}\\
    \applyone{\ph{}}{\sig{}}{\noeffect} & \bullet & \\
    \applyone{\t{\pi'}} {\sig{}} {\pi(\x{})} & \t{{\pi'(\pi(\x{}))}} & \\
    \applyone{\comp{\t{}}_{\pi'}} {\sig{}} {\pi(\x{})} & \comp{\t{}}_{{\pi'(\pi(\x{}))}} & \\
   \end{array}
$$

Cases 2 and 3 of {\applyone{-}{-}{-}} are redundant with the {\sc
  T-Bot} type rule.  Probably one one of them is needed.

\subsection{Combine}

There are many possibilities for $\combfilter{-}{-}{-}$.  One is the
very conservative

$$%
\begin{array}{l@{\ =\ }l@{\qquad}l}
\combfilter{\phii1}{\phii2}{\phii3} & \bullet|\bullet & 
\end{array}
$$

\noindent
A more sophisticated version is:


\newcommand{\ts}{\uniontwo{\t{}}{\s{}}}

$$
\begin{array}{l@{\ =\ }l@{\qquad}l}
\combfilter{\phii{}}{\bullet|\bot}{\bot|\bullet} & \phii{}
& \text{  the student expansion}\\

\combfilter{\bullet|\bot}{\phii2}{\phii3} & \phii2 & \\

\combfilter{\bot|\bullet}{\phii2}{\phii3} & \phii3 & \vspace{3mm}\\

\multicolumn{2}{l}{\combfilter 
{\t{\pi(\x{})}::\op{1_+}|\comp{\t{}}_{\pi(\x{})}::\op{1_-}}
{\bullet|\bot}
{{\sig{\pi(\x{})}::\op{3_+}|\comp{\sig{}}_{\pi(\x{})}::\op{3_-}}} 
}
  & \text{ or with predicates} \\
& \ts_{\pi(\x{})}::\op+|{\comp{\ts}}_{\pi(\x{})}::\op- & \\
\multicolumn{3}{c}{\text{where } \op+|\op- = \combfilter{\op1_+|\op1_-}{\bullet|\bot}{\op3_+|\op3_-}} \vspace{3mm}\\

\combfilter{\op{1_+}|\op{1_-}} {\bullet|\bot} {\op{3_+}|\op{3_-}} & {\bullet|\op{1_-},\op{3_-}}& \text{ or} \vspace{3mm}\\

\combfilter{\op{1_+}|\op{1_-}} {\op{2_+}|\op{2_-}} {\bot|\bullet} & \op{1_+},\op{2_+}|\bullet & \text{ and} \vspace{3mm}\\

\combfilter{\phii{}}{\phii{}'}{\phii{}'} & \phii{}' & \\

\combfilter{\phii1}{\phii2}{\phii3} & \bullet|\bullet & \text{ otherwise}
\end{array}
$$

Rules 2 and 3 might only be needed in the proof.


\newpage

\section{Operational Semantics}

\[
\inferrule*[lab={E-Delta}]
        {\del{\c{}}{\vv{}} = \vv{}'}
        {\step{\comb{\c{}}{\vv{}}}{\vv{}'}}
\qquad\qquad
\inferrule[E-Beta]{}
      {\step{\comb{\abs[i]{\x{}}{\t{}}{\e{b}}}{\movervec[i]{\e{a}}}}{\subsmulti{\e{b}}{\x{}}{\e{a}}{i}}}
\]

\[
\inferrule[E-IfFalse]{}
      {\step{\cond{\ff}{\e2}{\e3}}{\e3}}
\qquad\qquad
\inferrule[E-IfTrue]{\vv{} \neq \ff}
      {\step{\cond{\vv{}}{\e2}{\e3}}{\e2}}
\]

\[
\inferrule*
        {\step{L}{R}}
        {\reduce{\E{}[L]}{\E{}[R]}}
\]


\newpage

\section{Constants}


\[
\begin{array}{@{}l@{\ =\ }l@{\qquad}l}
  \dt{\numberp} & {\predty{\bullet}{\num}} & \\
  \dt{\procp} & {\predty{\bullet}{\proctop}} & \\
  \dt{\boolp} & {\predty{\bullet}{\bool}} & \\
  \dt{\consp} & {\predty{\bullet}{\consty{\top}{\top}}} & \\
  \dt{\addone} & {\simpleproc{\num}{\num}} & \\
  \dt{\notsym} & {\simpleproc{\top}{\bool}} & \\
  \dt{\zerop} & {\simpleproc{\num}{\bool}} & \\
\end{array}
\]

\vspace{5mm}

\[
\begin{array}{@{}l@{\ =\ }l@{\qquad}l}
  \del{\addone}{\n{}} & \n{} + 1 & \vspace{2mm}\\

  \del{\notsym}{\ff} & \tt & \\
  \del{\notsym}{\vv{}} & \ff & \text{ otherwise}\vspace{2mm}\\

  \del{\zerop}{0} & \tt & \\
  \del{\zerop}{\n{}} & \ff & \text{ otherwise} \vspace{2mm}\\

  \del{\numberp}{\n{}} & \tt & \\
  \del{\numberp}{\vv{}} & \ff & \text{ otherwise}\vspace{2mm}\\

  \del{\boolp}{\tt} & \tt & \\
  \del{\boolp}{\ff} & \tt & \\
  \del{\boolp}{\vv{}} & \ff & \text{ otherwise}\vspace{2mm}\\

  \del{\procp}{\abs{\x{}}{\t{}}{\e{}}} & \tt & \\
  \del{\procp}{\vv{}} & \ff & \text{ otherwise}\vspace{2mm}\\

  \del{\car}{\cons{\vv1}{\vv2}} & \vv1 & \\
  \del{\cdr}{\cons{\vv1}{\vv2}} & \vv2 & \\
\end{array}
\]

\newpage
\section{Environment Operations}

\newcommand{\update}[3][\Gamma(\x{})]{\mathrm{update}({#1},{#3}_{#2})}
\newcommand{\updatesimp}[2]{\mathrm{update}({#1},{#2})}

$$
\begin{array}{@{\Gamma\ +\ }l@{\ =\ }l@{\qquad}l}
  \t{\pi(\x{})},\op{} & (\Gamma,\x{}:\update{\pi}{\t{}}) + \op{}& \\
  \comp{\t{}}_{\pi(\x{})},\op{} & (\Gamma,\x{}:\update{\pi}{\comp{\t{}}}) + \op{}& \\
  \bot,\op{} & \Gamma' & \text{ where } {\forall \x{}}.\Gamma'(\x{}) = \bot\ \\ %
  \bullet & \Gamma & \\
%  \p{} & \Gamma,\x{}:\updatesimp{\Gamma(\x{})}{\p{}} & \\
%  \comp{\t{}}_{\pi(\x{})} & \Gamma,\x{}:\update{\pi}{\comp{\t{}}} & \\
%  \bot & \Gamma' & \text{ where } {\forall \x{}}.\Gamma'(\x{}) = \bot\ \\ %
\end{array}
$$

$$
\begin{array}{@{}l@{\ =\ }l@{\qquad}l}
  \update[\consty{\t{}}{\sig{}}]{\pi::\pecar}{\ups{}} & \consty{\update[\t{}]{\pi}{\ups{}}}{\sig{}}  & \\
  \update[\consty{\t{}}{\sig{}}]{\pi::\pecar}{\comp{\ups{}}} & \consty{\update[\t{}]{\pi}{\comp{\ups{}}}}{\sig{}}  & \\

  \update[\consty{\t{}}{\sig{}}]{\pi::\pecdr}{\ups{}} & \consty{\t{}}{\update[\sig{}]{\pi}{\ups{}}}  & \\
  \update[\consty{\t{}}{\sig{}}]{\pi::\pecdr}{\comp{\ups{}}} & \consty{\t{}}{\update[\sig{}]{\pi}{\comp{\ups{}}}}  & \\

  \update[\t{}]{\bullet}{\ups{}} & \restrict{\t{}}{\ups{}}& \\

  \update[\t{}]{\bullet}{\comp{\ups{}}} & \remove{\t{}}{\ups{}}& \\
%  \updatesimp{\t{}}{\both} & {\bot} & \\
\end{array}
%
$$

\vspace{5mm}
$$
\begin{array}{@{}l@{\ =\ }l@{\qquad}l}
  \restrict{\t{}}{\sig{}} & \bot & \text{ if } {\nooverlap{\t{}}{\sig{}}} \\
  \restrict{(\usym\ \overvec{\t{}})}{\sig{}} & (\usym\ \overvec{\restrict{\t{}}{\sig{}}})& \\
  \restrict{\t{}}{\sig{}} & \t{} & \text{ if } {\subtype{\t{}}{\sig{}}} \\
  \restrict{\t{}}{\sig{}} & \sig{} & \text{ otherwise} \vspace{5mm}\\


  \remove{\t{}}{\sig{}} & \bot & \text{ if } {\subtype{\t{}}{\sig{}}} \\
  \remove{(\usym\ \overvec{\t{}})}{\sig{}} & (\usym\ \overvec{\remove{\t{}}{\sig{}}})& \\
  \remove{\t{}}{\sig{}} & \t{} & \text{ otherwise} \vspace{5mm}\\

  \nooverlap{\num}{\bool} & \mathrm{true} & \\
  \nooverlap{\num}{\consty{\t{}}{\sig{}}} & \mathrm{true} & \\
  \nooverlap{\bool}{\consty{\t{}}{\sig{}}} & \mathrm{true} & \\
  \nooverlap{\num}{\proctype{\movervec[i]{\t{}}}{\sig{}}{\movervec[i]{\phih{}}}{\sh{}}} & \mathrm{true} & \\
  \nooverlap{\bool}{\proctype{\movervec[i]{\t{}}}{\sig{}}{\movervec[i]{\phih{}}}{\sh{}}} & \mathrm{true} & \\
  \nooverlap
  {\consty{\t{}}{\sig{}}}
  {\proctype{\movervec[i]{\ups{a}}}{\ups{r}}{\movervec[i]{\phih{}}}{\sh{}}} & 
  \mathrm{true} & \\
  \nooverlap{(\usym\ \overvec{\t{}})}{\sig{}} & \bigwedge
  \overvec{\nooverlap{\t{}}{\sig{}}} & \\
  \nooverlap{\t{}}{\sig{}} & \mathrm{true} & \text{ if } \nooverlap{\sig{}}{\t{}}\\
  \nooverlap{\t{}}{\sig{}} & \mathrm{false} & \text{ otherwise}\\
\end{array}
$$

\newpage
\section{Examples}

\subsection{Expression Examples}

\newcommand{\nump}[1]{\ma{\comb{\numberp}{#1}}}

\newcommand{\carx}{\ma{\comb{\car}{\x{ }}}}

\newcommand{\pecarx}{\ma{\pecar(\x{})}}
\newcommand{\pecarcarx}{\ma{\pecar(\pecar(\x{}))}}

\newcommand{\xenv}[1]{\x{}:{#1}}
\newcommand{\xenvtop}{\xenv{\top}}
\newcommand{\xyenv}{\xenvtop,\y{}:\top}


\newcommand{\numpx}{\nump{\x{}}}
\newcommand{\boolpx}{\comb{\boolp}{\x{}}}
\newcommand{\boolpy}{\comb{\boolp}{\y{}}}

\newcommand{\cand}[2]{\comb{\andsym}{{#1}\ {#2}}}
\newcommand{\cor}[2]{\comb{\orsym}{{#1}\ {#2}}}

\newcommand{\numbool}{{(\usym\ \num\ \bool)}}
\newcommand{\inumbool}{{(\isym\ \num\ \bool)}}


\begin{displaymath}
  \hastyvar[\xenvtop]{\x{}}{\top}{\x{}}
\end{displaymath}

\begin{displaymath}
  \hastyvar[{\xenv{\consty{\top}{\top}}{}}]{\carx{}}{\top}{\pecarx{}}
\end{displaymath}

\begin{displaymath}
  \hastypred [\xenvtop] {\numpx} {\num} {\x{}}
\end{displaymath}

\begin{displaymath}
  \hastypred [\xenv{\consty{\top}{\top}}] {\nump{\carx}} {\num} {\pecarx}
\end{displaymath}

\begin{displaymath}
  \hastyfalse[\empty]{\ff}{\fft}
\end{displaymath}

\begin{displaymath}
  \hastytrue[\empty]{3}{\num}
\end{displaymath}

\begin{displaymath}
  \hastyeff[\xyenv] {\cand{\numpx}{\boolpy}} {\bool} {\num_{x{}},\bool_{\y{}}}  {\bullet} {\noeffect}
\end{displaymath}

\begin{displaymath}
  \hastyeff[\xyenv] {\cor{\numpx}{\boolpy}} {\bool} {\bullet} {\comp{\num}_{x{}},\comp{\bool}_{\y{}}}  {\noeffect}
\end{displaymath}

\begin{displaymath}
  \hastyeff[\xenvtop] {\cand{\numpx}{\boolpx}} {\bool} {\num_{x{}},\bool_{\x{}}}  {\bullet} {\noeffect}
\end{displaymath}

\begin{displaymath}
  \hastypred[\xenvtop] {\cor{\numpx}{\boolpx}} {\numbool} {\x{}}
\end{displaymath}

or alternatively, if we had intersection types:

\begin{displaymath}
  \hastypred[\xenvtop] {\cand{\numpx}{\boolpx}} {\inumbool} {\x{}}
\end{displaymath}

\newpage

\subsection{Function Examples}

\newcommand{\consab}{\consty{\a{}}{\b{}}}

\begin{displaymath}
  \hastytrue {\car} {\latentvarty{\consab}{\a{}}{\pecar}{\pecar(0)}}
\end{displaymath}

\begin{displaymath}
  \hastytrue{\abssingle{\x{}}{\a{}}{\x{}}} {\latentvarty{\a{}}{\a{}}{\bullet}{0}}
\end{displaymath}

\begin{displaymath}
  \hastytrue {\numberp} {\predty{\bullet}{\num}}
\end{displaymath}

\begin{displaymath}
  \hastytrue {\abssingle{\x{}}{\top}{\nump{\x{}}}} {\predty{\bullet}{\num}}
\end{displaymath}

\begin{displaymath}
  \hastytrue {\abssingle{\x{}}{\consab}{\nump{\carx}}} {\predty[{\consab}]{\pecar}{\num}}
\end{displaymath}

\begin{displaymath}
  \hastytrue {\abssingle{\x{}} {\consab} {\nott{\nump{\carx}}}}
  {\proctype{{\consab}}{\bool}{{\comp{\num}_{\pecar}|{{\num}}_{\pecar}}}{\noeffect}}
\end{displaymath}

\begin{displaymath}
  \hastytrue {\abssingle{\x{}}{\a{}}{\ff}} {\proctype{\a{}}{\bool}{\both|\bullet}{\noeffect}}
\end{displaymath}


\ifmarg
\begin{displaymath}
  \hastytrue
  {\absgeneral {\x{} : \top, \y{} : \top} {\nump{\x{}}}} 
  {\proctype{\top, \top}{\bool}{{\num}_{\bullet}|{\comp{\num}}_{\bullet},\bullet|\bullet}{\noeffect}}
\end{displaymath}
\else
\fi

\newpage

\section{Larger Example}

\def\prty{\consty{\consty{\top}{\top}}{\top}}
\def\prtyf{\consty{\consty{\fft}{\top}}{\top}}
\def\env{\x{}:\prty}
\def\app{\comb{\car}{\carx}}
\def\concl{\hastyeffphi{\cond{\app}{\tt}{\app}}{\bool}{\bullet|\bullet}{\noeffect}}



Let $\Gamma = \env$.

\vspace{1cm}

%begin{turn}{270}

\small
$
\inferrule*
{
  \framebox{\(\inferrule*{
      \inferrule*{\hastyvar{\x{}}{\prty}{\x{}}}{\hastyvar{\carx}{\consty{\top}{\top}}{\pecarx}}}
    {\hastyvar{\app}{\top}{\pecarcarx}}\)
  }
\and
\framebox{\(
\inferrule*{
  {\framebox{\(
      \inferrule*
      {\hastytrue[{\x{}:\prtyf}]{\x{}}{\prtyf}}
      {\hastytrue[{\x{}:\prtyf}]{\carx}{\consty{\top}{\fft}}} \)}}\\\\
  \Gamma + {{\fft}_{\pecarcarx}} = {\x{}:\prtyf} }
 {\hastyfalse[\Gamma + {{\fft}}_{\pecarcarx}]{\app}{\fft}}\)} \\
 \hastytrue[\Gamma + \comp{\fft}_{\pecarcarx}]{\tt}{\ttt} \\
 {\bullet|\bullet} = \combfilter{\fft_{\pecarcarx}|{\comp{\fft}}_{\pecarcarx}}{\bullet|\bot}{\bot|\bullet}}
{\concl}
$
%end{turn}

\normalsize

\newpage

\section{Theorems and Proofs}

\newometa\opp{\psi'}

\subsection{Preliminary Definitions}

\[
\inferrule{
\forall \G{}. \forall \x{} \in \mbox{dom}(\G{}). \subtype{(\G{} + \opp{})(x)}{(\G{} + \op{})(x)}
}
{
  \vdash \op{} < \opp{}
}
\]

\[
\consty{\t{}}{\sig{}} @ \pecar :: \pi = \t{} @ \pi
\qquad
\consty{\t{}}{\sig{}} @ \pecdr :: \pi = \sig{} @ \pi
\qquad
\t{} @ \epsilon = \t{}
\]

\[
\hastype{\x{}}{\t{}} \models \bot
\qquad
\inferrule{\subtype{\t{}' @ \pi}{\t{}}}
{\hastype{\x{}}{\t{}'} \models \t{\pi(\x{})}}
\qquad
\inferrule{\notsubtype{\t{}' @ \pi}{\t{}}}
{\hastype{\x{}}{\t{}'} \models \compt{\pi(\x{})}}
\]
\[
\inferrule{\x{} \neq \y{}}
{\hastype{\x{}}{\t{}'} \models \t{\pi(\y{})}}
\qquad
\inferrule{\x{} \neq \y{}}
{\hastype{\x{}}{\t{}'} \models \compt{\pi(\y{})}}
\]

Note that we can use $\notsubtype{\t{}' @ \pi}{\t{}}$ instead of the
``overlap'' relation since everything is a value.

\subsection{Substitution Lemma}

If \hastyeffphi [\G{},{\movervec[i]{\hastype{\x{}}{\sig{}}}}] {\e{}} {\t{}} {\op{+} | \op{-}} {\s{}}

\noindent
and \hastyeffphi [] {\vv{}} {\sig{}'} {\phii{0}} {\s{0}}

\noindent
and \subtype{\sig{}'}{\sig{}}

\noindent
Then $\hastyeffphi {\subs{\e{}} {\x{}} {\vv{}}} {\t{}'} {\opp{+} | \opp{-}} {\s{}'}$

\noindent
and \subtype{\t{}'}{\t{}}

\noindent
and $\hastype{\x{}}{\sig{}'} \models \op{\pm} \quad \Rightarrow \quad \vdash \opp{\pm} < \op{\pm}$


\noindent
and $\hastype{\x{}}{\sig{}'} \not\models \op{\pm} \quad \Rightarrow \quad \opp{\pm} = \bot$

\noindent
and either $\s{} = \pi(\x{}) \wedge \s{}' = \noeffect$ or $\s{} = \s{}'$
\end{document}

